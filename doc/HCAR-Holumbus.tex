\begin{hcarentry}{Holumbus Search Engine Framework}
\label{holumbus}
\report{Uwe Schmidt}
\status{first beta release}
\participants{Timo~B.~H\"ubel, Sebastian Schlatt, Stefan Schmidt}
\entry{new}
\makeheader

\subsubsection*{Description}

The Holumbus framework consists of a set of modules and tools
for creating fast, flexible and highly customizable search engines with Haskell.
The framework consists of two main parts. The first part is the indexer for extracting the data
of a given type of documents, e.g. documents of a web site, and store it in an appropriate index.
The second part is the search engine for querying the index.

An instance of the Holumbus framework is the Haskell API search engine Hayoo!
(\uri{http://holumbus.fh-wedel.de/hayoo/}. The web interface for Hayoo! is
implemented with the Janus web server, written in Haskell and based on HXT~\cref{hxt}.

\subsubsection*{Features}

\begin{compactitem}
\item Highly configurable crawler module for flexible indexing of structured data
\item Customizable index structure for an effective search
\item {\em find as you type} search
\item Suggestions
\item Fuzzy queries
\item Customizable result ranking
\item Index structure designed for distributed search
\item darcs repository with current development version under
  \url{http://darcs.fh-wedel.de/holumbus}
\end{compactitem}

\subsubsection*{Current Work}

Currently the indexer module will further be developed and extended,
such that the configuration about the relevant information especially
in web pages to be indexed becomes simple and easy. During this activity
new use cases of the framework will be implemented.

In another Master Thesis the distributed query evaluation in a network of
machines is tackled. Here we will try to adopt ideas from the Google
{\em map--reduce} approach for Holumbus. The aim is, to
develop a generally applicable map--reduce like framework and
take the Holumbus search and index manipulation as a serious test case.


\FurtherReading

The Holumbus web page
(\url{http://holumbus.fh-wedel.de/)
includes downloads, darcs web interface, current status, requirements
and documentation.
Timo H\"�bels Master Thesis describing the Holumbus index structure and
the search engine is avaliable at
\url{http://holumbus.fh-wedel.de/branches/develop/doc/thesis-searching.pdf}.

\end{hcarentry}
