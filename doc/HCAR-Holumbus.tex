\begin{hcarentry}[updated]{Holumbus Search Engine Framework}
\label{holumbus}
\report{Uwe Schmidt}%05/09
\participants{Timo~B.\ H\"ubel, Sebastian Reese, Sebastian Schlatt,
  Stefan Schmidt, Bj\"orn Peem\"oller, Stefan Roggensack
}
\status{first release}
\makeheader

\subsubsection*{Description}

The Holumbus framework consists of a set of modules and tools
for creating fast, flexible, and highly customizable search engines with Haskell.
The framework consists of two main parts. The first part is the indexer for extracting the data
of a given type of documents, e.g., documents of a web site, and store it in an appropriate index.
The second part is the search engine for querying the index.

An instance of the Holumbus framework is the Haskell API search engine Hayoo!\
(\url{http://holumbus.fh-wedel.de/hayoo/}). The web interface for Hayoo!\ is
implemented with the Janus web server, written in Haskell and based on HXT~\cref{hxt}.

The framework supports distributed computations for building indexes
and searching indexes. This is done with a MapReduce like framework.
The MapReduce framework is independent of the index- and
search-components, so it can be used to develop distributed systems
with Haskell.

The framework is now separated into four packages, all available on
Hackage.

\begin{itemize}
\item The Holumbus Search Engine 
\item The Holumbus Distribution Library
\item The Holumbus Storage System
\item The Holumbus MapReduce Framework
\end{itemize}

The search engine package includes the indexer and search modules,
the MapReduce package bundles the distributed MapReduce system.
This is based on two other packages, which may be useful for their on:
The Distributed Library with a message passing communication layer
and a distributed storage system.

\subsubsection*{Features}

\begin{itemize}
\item Highly configurable crawler module for flexible indexing of structured data
\item Customizable index structure for an effective search
\item {\em find as you type} search
\item Suggestions
\item Fuzzy queries
\item Customizable result ranking
\item Index structure designed for distributed search
\item Darcs repository with current development version under
  \url{http://darcs2.fh-wedel.de/holumbus}
\item Distributed building of search indexes
\end{itemize}

\subsubsection*{Current Work}

The indexer and search module will be used and extended
to support the Hayoo!\ engine for searching the hackage package library
(http://holumbus.fh-wedel.de/hayoo/hayoo.html).

Stefan Schmidt has finished his master thesis developing
the Holumbus MapReduce system,  a
framework for distributed computing with an architecture
like the Google
{\em map--reduce} system.

A follow-up thesis bringing this
MapReduce system into a development status for real world
distributed applications has been started by Sebastian Reese.
One subgoal of this work is to write a {\em cookbook}
for programming with the MapReduce framework and for giving
tuning and configuration hints.
The distributed recomputation and update of the Hayoo!\ index will
be one of the real world test cases of this project.

\FurtherReading

The Holumbus web page
(\url{http://holumbus.fh-wedel.de/})
includes downloads, Darcs web interface, current status, requirements, 
and documentation.
Timo H\"ubel's Master Thesis describing the Holumbus index structure and
the search engine is available at
\url{http://holumbus.fh-wedel.de/branches/develop/doc/thesis-searching.pdf}.
Sebastian Schlatt's thesis dealing with the crawler component is
available at
\url{http://holumbus.fh-wedel.de/src/doc/thesis-indexing.pdf}
The thesis of Stefan Schmidt describing the Holumbus MapReduce is
available via \url{http://holumbus.fh-wedel.de/src/doc/thesis-mapreduce.pdf}
\end{hcarentry}
