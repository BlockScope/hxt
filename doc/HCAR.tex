\begin{hcarentry}[updated]{Haskell XML Toolbox}
\label{hxt}
\report{Uwe Schmidt}
\status{sixed major release (current release: 7.1)}
\entry{updated}% done, 30.4.2007
\makeheader

\subsubsection*{Description}

The Haskell XML Toolbox is a collection of tools for processing XML with
Haskell. It is itself purely written in Haskell 98. The core component of the
Haskell XML Toolbox is a validating XML-Parser that supports
almost fully the Extensible Markup Language (XML) 1.0 (Second Edition),
There is a validator based on DTDs and a new more powerful validator for
Relax NG schemas.

The Haskell XML Toolbox bases on the ideas of HaXml~\cref{haxml} and HXML,
but introduces a more general approach for processing XML with Haskell.
Since release 5.1 there is a new arrow interface similar to the approach
taken by HXML. This interface is more flexible than the old filter approach.
It is also safer, type checking of combinators becomes possible with the arrow
interface.

\subsubsection*{Features}

\begin{compactitem}
\item Validating XML parser
\item Very liberal HTML parser
\item XPath support
\item Full Unicode support
\item Support for XML namespaces
\item Flexible arrow interface with type classes for XML filter
\item Package support for ghc
\item Native Haskell support of HTTP 1.1 and FILE protocol
\item HTTP and access via other protocols via external program curl
\item Tested with W3C XML validation suite
\item Example programs for filter and arrow interface
\item Relax NG schema validator based on the arrows interface
\item A HXT Cookbook for using the toolbox and the arrow interface
\item Basic XSLT support
\item darcs repository with current development version (7.2) under {\tt http://darcs.fh-wedel.de/hxt}
\end{compactitem}

\subsubsection*{Current Work}

A master thesis has been finished developing an XSLT
system. The result is a rather complete implementation of
an XSLT transformer system. Only minor features are missing.
The implementation consists of about only 2000 lines of Haskell code.
The XSLT module is included since the HXT 7.0 release.

A second master students project, the development of a web server called Janus,
has been finished in October of 2006.
The title is {\em A Dynamic Webserver with Servlet Functionality in
  Haskell Representing all Internal Data by Means of XML}.
HXT with the arrows interface has been used for processing all internal data of this web server.
The Janus server is highly configurable and can be used not only as HTTP server, but for
various other server like tasks.
The results of this work will be available via a darcs repository in June 2007.
Current activity consists of testing, example applications, demos and documentation.

A new project, an application for HXT and Janus will start in summer 2007:
Two master students will construct an index and search engine for specialized
search tasks. This system will be highly configurable, such that tasks like
searching within a web site, search of articles within a book store or
search within a newspaper archive becomes possible. Distribution of the index
and search engines within a network architecture will be an additional aspect
of this project.

\FurtherReading

The Haskell XML Toolbox Web page
(\url{http://www.fh-wedel.de/~si/HXmlToolbox/index.html})
includes downloads, online API documentation, a cookbook with nontrivial examples
of XML processing using arrows and RDF documents, and master thesises describing the
design of the toolbox, the DTD validator, the arrow based Relax NG
validator and the XSLT system.
A getting started tutorial about HXT is avaliable in the Haskell Wiki (\url{http://www.haskell.org/haskellwiki/HXT}).
\end{hcarentry}
