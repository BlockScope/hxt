\begin{hcarentry}{Haskell XML Toolbox}
\label{hxt}
\report{Uwe Schmidt}
\status{seventh major release (current release: 8.0.0)}
\participants{(Christial Uhlig)}
\entry{changed}
\makeheader

\subsubsection*{Description}

The Haskell XML Toolbox is a collection of tools for processing XML with
Haskell. It is itself purely written in Haskell 98. The core component of the
Haskell XML Toolbox is a validating XML-Parser that supports
almost fully the Extensible Markup Language (XML) 1.0 (Second Edition),
There is a validator based on DTDs and a new more powerful one for
Relax NG schemas.

The Haskell XML Toolbox bases on the ideas of HaXml~\cref{haxml} and HXML,
but introduces a more general approach for processing XML with Haskell.
The processing model is based on arrows. The arrow interface is more flexible
than the filter approach taken in the earlier HXT versions and in HaXml.
It is also safer, type checking of combinators becomes possible with the arrow
approach.

\subsubsection*{Features}

\begin{compactitem}
\item Validating XML parser
\item Very liberal HTML parser
\item Lightweight lazy parser for XML/HTML based on Tagsoup
\item Easy de-/serialization between native Haskell data and XML by pickler and pickler combinators
\item XPath support
\item Full Unicode support
\item Support for XML namespaces
\item Cabal package support for ghc
\item Native Haskell support of HTTP 1.1 and FILE protocol
\item HTTP and access via other protocols via external program curl
\item Tested with W3C XML validation suite
\item Example programs
\item Relax NG schema validator
\item A HXT Cookbook for using the toolbox and the arrow interface
\item Basic XSLT support
\item darcs repository with current development version (8.0.1) under
  \url{http://darcs.fh-wedel.de/hxt}
\end{compactitem}

\subsubsection*{Current Work}

In a master student's project done by Christian Uhlig, the development of a web server called Janus,
has been finished.
The title is \emph{A Dynamic Webserver with Servlet Functionality in
  Haskell Representing all Internal Data by Means of XML}.
HXT has been used for processing all internal data of this web server.
The Janus server is highly configurable and can be used not only as HTTP server, but for
various other server like tasks.
This server is used and will be further developed and extended within another
project called Holumbus (\url{http://holumbus.fh-wedel.de/}.
Holumbus is a framework for developing specialized
search engines.
The Janus system is available via \url{http://darcs.fh-wedel.de/janus}.
Current activity consists of testing, example applications, demos and documentation.
An application of Janus is the Hayoo! search engine (\uri{http://holumbus.fh-wedel.de/hayoo/}
implemented with the Holumbus framework~\cref{holumbus}.

\FurtherReading

The Haskell XML Toolbox Web page
(\url{http://www.fh-wedel.de/~si/HXmlToolbox/index.html})
includes downloads, online API documentation, a cookbook with nontrivial examples
of XML processing using arrows and RDF documents, and master thesises describing the
design of the toolbox, the DTD validator, the arrow based Relax NG
validator and the XSLT system.
A getting started tutorial about HXT is avaliable in the Haskell Wiki (\url{http://www.haskell.org/haskellwiki/HXT}
).
\end{hcarentry}
