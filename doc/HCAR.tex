\begin{hcarentry}[updated]{Haskell XML Toolbox}
\label{hxt}
\report{Uwe Schmidt}%11/08
\status{seventh major release (current release: 8.3.0)}
\makeheader

\subsubsection*{Description}

The Haskell XML Toolbox (HXT) is a collection of tools for processing XML with
Haskell. It is itself purely written in Haskell 98. The core component of the
Haskell XML Toolbox is a validating XML-Parser that supports
almost fully the Extensible Markup Language (XML) 1.0 (Second Edition).
There is a validator based on DTDs and a new more powerful one for
Relax~NG schemas.

The Haskell XML Toolbox is based on the ideas of HaXml~\cref{haxml} and HXML,
but introduces a more general approach for processing XML with Haskell.
The processing model is based on arrows. The arrow interface is more flexible
than the filter approach taken in the earlier HXT versions and in HaXml.
It is also safer; type checking of combinators becomes possible with the arrow
approach.

HXT consists of two packages, the old first approach (hxt-filter)
based on filters and the newer and more flexible and save approach
using arrows (hxt). The old package hxt-filter, will further be maintained to work
with the latest ghc version, but new development will only be done
with the arrow based hxt package.

\subsubsection*{Features}

\begin{itemize}
\item Validating XML parser
\item Very liberal HTML parser
\item Lightweight lazy parser for XML/HTML based on Tagsoup~\cref{tagsoup}
\item Easy de-/serialization between native Haskell data and XML by pickler and pickler combinators
\item XPath support
\item Full Unicode support
\item Support for XML namespaces
\item Cabal package support for GHC
\item HTTP access via Haskell bindings to libcurl
\item Tested with W3C XML validation suite
\item Example programs
\item Relax~NG schema validator
\item An HXT Cookbook for using the toolbox and the arrow interface
\item Basic XSLT support
\item Darcs repository with current development version (8.3.1) under
  \url{http://darcs2.fh-wedel.de/hxt}
\end{itemize}

\subsubsection*{Current Work}

Currently mainly maintenance work is done. This includes
space and runtime optimizations, the internal representation of XML
names has been changed to gain less memory consumption. Equal XML names
share the same main memory.

It is planned to further develop and extend the validation part with
Relax~NG and the conversion from/to Haskell internal data. The pickler
approach used in that task can be extended to derive DTDs, Relax~NG
Schemas or XML Schemas for Validation of the external XML representation.

The HXT library is extensively used in the Holumbus
project~\cref{holumbus}, there it forms the basis for the index generation.

\FurtherReading
The Haskell XML Toolbox Web page
(\url{http://www.fh-wedel.de/~si/HXmlToolbox/index.html})
includes downloads, online API documentation, a cookbook with nontrivial examples
of XML processing using arrows and RDF documents, and master theses describing the
design of the toolbox, the DTD validator, the arrow based Relax~NG
validator, and the XSLT system.

A getting started tutorial about HXT is available in the Haskell Wiki (\url{http://www.haskell.org/haskellwiki/HXT}
). The conversion between XML and native Haskell datatypes is
described in another Wiki page
(\url{http://www.haskell.org/haskellwiki/HXT/Conversion_of_Haskell_data_from/to_XML}).
\end{hcarentry}
