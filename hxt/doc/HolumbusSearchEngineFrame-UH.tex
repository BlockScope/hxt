\begin{hcarentry}[updated]{Holumbus Search Engine Framework}
\label{holumbus}
\report{Uwe Schmidt}%05/10
\participants{Timo~B.\ H\"ubel, Sebastian Gauck,
  Stefan Schmidt, Bj\"orn Peem\"oller, Stefan Roggensack, Sebastian Reese, Alexander Treptow, Uwe Schmidt}
\status{first release}
\makeheader

\subsubsection*{Description}

The Holumbus framework consists of a set of modules and tools
for creating fast, flexible, and highly customizable search engines with Haskell.
The framework consists of two main parts. The first part is the indexer for extracting the data
of a given type of documents, e.g., documents of a web site, and store it in an appropriate index.
The second part is the search engine for querying the index.

An instance of the Holumbus framework is the Haskell API search engine Hayoo!\
(\url{http://holumbus.fh-wedel.de/hayoo/}). The web interface for Hayoo!\ is
implemented with the Janus web server, written in Haskell and based on HXT~\cref{hxt}.

The framework supports distributed computations for building indexes
and searching indexes. This is done with a MapReduce like framework.
The MapReduce framework is independent of the index- and
search-components, so it can be used to develop distributed systems
with Haskell.

The framework is now separated into four packages, all available on
Hackage.

\begin{itemize}
\item The Holumbus Search Engine 
\item The Holumbus Distribution Library
\item The Holumbus Storage System
\item The Holumbus MapReduce Framework
\end{itemize}

The search engine package includes the indexer and search modules,
the MapReduce package bundles the distributed MapReduce system.
This is based on two other packages, which may be useful for their on:
The Distributed Library with a message passing communication layer
and a distributed storage system.

\subsubsection*{Features}

\begin{itemize}
\item Highly configurable crawler module for flexible indexing of structured data
\item Customizable index structure for an effective search
\item {\em find as you type} search
\item Suggestions
\item Fuzzy queries
\item Customizable result ranking
\item Index structure designed for distributed search
\item Git repository containing the current development version of all packages under
  \url{http://holumbus.fh-wedel.de/src.git}
\item Distributed building of search indexes
\end{itemize}

\subsubsection*{Current Work}

The data structures of the Holumbus indexes have been optimized
for space and time. There is a new and efficient prefix tree structure,
which further enables index updates.

The indexer and search module is used
to support the Hayoo!\ engine for searching the hackage package library
(http://holumbus.fh-wedel.de/hayoo/hayoo.html). Because of the fast growing number
of packages on hackage, the Hayoo! search engine will be extended by a package search.

Sebastian Reese has finished his work on applying the
the MapReduce framework and for giving
tuning and configuration hints. Benchmarks for various small problems
and for generating search indexes have shown, that the architecture
scales very well.

In the subproject of Holumbus, the so called Hawk framework, 
Bj\"orn Peem\"oller and Stefan Roggensack have developed a web framework
for Haskell. Currently Alexander Treptow is applying, testing and extending the
framework. A first application is a customizable search for Hayoo!

The Hawk system is comparable in functionality and architecture with Ruby on Rail
and other web frameworks. Its architecture follows the MVC pattern.
It consists of a simple relational database mapper for persistent storage of data
and a template system for the view component. This template system has two
interesting features: Fist the templates are valid XHTML documents. The parts,
where data has to be filled in, are marked with Hawk specific elements and attributes.
These parts are in a different namespace, so they do not destroy the XHTML structure.
The second interesting feature is, that the templates contain type descriptions for
the values to be filled in. This type information enables a static type check, whether
the models and views fit together.

The Hawk framework is independent of the Holumbus search engine.
It will be applicable for the development of arbitrary web applications.

\FurtherReading

The Holumbus web page
(\url{http://holumbus.fh-wedel.de/})
includes downloads, Git web interface, current status, requirements, 
and documentation.
Timo H\"ubel's master thesis describing the Holumbus index structure and
the search engine is available at
\url{http://holumbus.fh-wedel.de/branches/develop/doc/thesis-searching.pdf}.
Sebastian Gauck's  thesis dealing with the crawler component is
available at
\url{http://holumbus.fh-wedel.de/src/doc/thesis-indexing.pdf}
The thesis of Stefan Schmidt describing the Holumbus MapReduce is
available via \url{http://holumbus.fh-wedel.de/src/doc/thesis-mapreduce.pdf}
\end{hcarentry}
