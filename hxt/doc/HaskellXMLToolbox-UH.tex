% HaskellXMLToolbox-UH.tex
\begin{hcarentry}[updated]{Haskell XML Toolbox}
\label{hxt}
\report{Uwe Schmidt}%05/11
\status{seventh major release (current release: 9.1)}
\makeheader

\subsubsection*{Description}

The Haskell XML Toolbox (HXT) is a collection of tools for processing XML with
Haskell. It is itself purely written in Haskell 98. The core component of the
Haskell XML Toolbox is a validating XML-Parser that supports
almost fully the Extensible Markup Language (XML) 1.0 (Second Edition).
There is a validator based on DTDs and a new more powerful one for
Relax~NG schemas.

The Haskell XML Toolbox is based on the ideas of HaXml %(\url{http://haskell.org/communities/05-2009/html/report.html#sect5.13.2})
and HXML,
but introduces a more general approach for processing XML with Haskell.
The processing model is based on arrows. The arrow interface is more flexible
than the filter approach taken in the earlier HXT versions and in HaXml.
It is also safer; type checking of combinators becomes possible with the arrow
approach.

HXT is partitioned into a collection of smaller packages: The core
package is
{\tt hxt}. It contains a validating XML parser, an HTML parser,
filters for manipulating XML/HTML and so called XML pickler for
converting XML to and from native Haskell data.

Basic functionality for character handling and decoding is
separated into the packages {\tt hxt-charproperties} and {\tt
 hxt-unicode}. These packages may be generally useful even for non XML projects.

HTTP access can be done with the help of the packages
{\tt hxt-http} for native Haskell HTTP access and {\tt hxt-curl} via a
libcurl binding. An alternative lazy non validating parser for XML and HTML can be
found in {\tt hxt-tagsoup}. 

The XPath interpreter is in package {\tt hxt-xpath}, the XSLT part in
{\tt hxt-xslt}
and the Relax~NG validator in {\tt hxt-relaxng}. For checking the XML
Schema Datatype definitions, also used with Relax~NG, there is a
separate and generally useful regex package {\tt hxt-regex-xmlschema}.

The old HXT approach working with filter {\tt hxt-filter} is still
available,
but currently only with hxt-8. It has not (yet) been updated to the
hxt-9 mayor version.

\subsubsection*{Features}

\begin{compactitem}
\item Validating XML parser
\item Very liberal HTML parser
\item Lightweight lazy parser for XML/HTML based on Tagsoup~\cref{tagsoup}
\item Binding to the expat parser via hexpat package
\item Easy de-/serialization between native Haskell data and XML by pickler and pickler combinators
\item XPath support
\item Full Unicode support
\item Support for XML namespaces
\item Cabal package support for GHC
\item HTTP access via Haskell bindings to libcurl and via Haskell HTTP
  package
\item Tested with W3C XML validation suite
\item Example programs
\item Relax~NG schema validator
\item Lightweight regex library with full support of Unicode and XML Schema
  Datatype regular expression syntax
\item An HXT Cookbook for using the toolbox and the arrow interface
\item Basic XSLT support
\item GitHub repository with current development versions of all packages
  \url{http://github.com/UweSchmidt/hxt}
\end{compactitem}

\subsubsection*{Current Work}

Besides maintenance work, there were some activities for
better IO and parser performance. The native XML as well as the HTML parser have
been optimized for speed and space. The input and output routines now work
with bytestrings instead of native Haskell IO. Furthermore the XPath component
has internally been changed for better performance, especially for the
handling of XPath node sets.

There are some plans to further develop the Relax~NG validator
for full XML Schema Datatype support and for the native Relax~NG
schema notation. Another topic in this field is the (semi-)automatic
Haskell datatype derivation out of Relax~NG schemas and the generation
of picklers between the schema and the Haskell types.

\FurtherReading
The Haskell XML Toolbox Web page
(\url{http://www.fh-wedel.de/~si/HXmlToolbox/index.html})
includes links to downloads,  documentation, and further information.

A getting started tutorial about HXT is available
 in the Haskell Wiki (\url{http://www.haskell.org/haskellwiki/HXT}
). The conversion between XML and native Haskell data types is
described in another Wiki page
(\url{http://www.haskell.org/haskellwiki/HXT/Conversion_of_Haskell_data_from/to_XML}).
\end{hcarentry}
